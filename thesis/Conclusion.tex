\chapter{Conclusion}\thispagestyle{fancy}

\paragraph*{}
The random Boolean networks, especially the $N$-$K$ models, are interesting models to study due to their definition's simplexity while showing a complex variety of different dynamics. They are a prototypical dynamical system and have many observable features, like attractors or phase transitions for a variation of certain parameters.

\paragraph*{}
In many dynamical systems in which we differentiate between separated phases, there is a trichotomy between a high degree of order, very chaotic or even random formations and an in-between, which often gets called critical or complex. The random Boolean networks are no exclusion from that, as we have seen ourselves. 

\paragraph*{}
In the frozen phase, they have this high ordering in a sense where there is barely anything going on from the dynamical perspective. This got very clear when we looked at their G-densities. It is the highest compared to other networks, which shows that most of the paths that lead to an attractor are very short. Moreover, these attractors are then also almost always either a fixpoint or a cycle of length two. Having them named frozen seems to be an accurate choice.

\paragraph*{}

The behavior of a chaotic network is quite different from the frozen systems. Here the paths to the attractors can get fairly long. The number of garden-of-Eden states is comparably the lowest and the cycle lengths can get long as well. As we have seen, the best fitting description for them is the annealed approximation. We analyze them in terms of a random walk over finitely large and discrete phase space is for all cases we have seen pretty accurate. It helps predict a few things like the probability distribution of attractors with a certain cycle length. It can be used to explain the transition behavior, as we have seen in the previous section. One clear rule to state that applies to the chaotic networks would be that the more attractors we have in a single realization, the less stable each of them will be.

\paragraph*{}
Also, the sizes of the averages of the basins of attraction have a monotonically increasing functional dependency on the cycle lengths. Again, using the picture of a random walk, it is absolutely evident that if we have several attractors and want to connect all the other states of the phase space in a random way to them, then the probability of connecting to the largest one or one of the states in its basin of attraction will also be the highest.

\paragraph*{}
As we have seen, the number of attractors and the cycles' lengths are closely related to the average of the size of the basins of attraction. Especially for critical networks, we had to look at them together to understand their variations for different lengths. As we have seen, we can assume there exist independent subsystems for many random Boolean networks that create longer cycles. The least common multiple of the lengths of the subattractors determines the number of states in the long total cycle.

\section{Further Research}
\paragraph*{}
There are still open problems that could be of interest for further research. The first one would be to look for the independent subsystems to explain the mean values of the basins of attraction. We have strong reasons to believe that they exist and play a role in constructing systems with long attractors, but to be certain, they would need to be studied directly within the model. This could be done in one of the following ways: One could look at the periodicities of the Boolean values of every node in a cycle or look at the networks' topologies directly and combine this with the acknowledgment of constant and fully canalizing functions. A combination of these two approaches is also thinkable.

\paragraph*{}
Another possible way to go further could be to look rather in the way of attractor robustness. We divided the nodes into the frozen and active (not-frozen) ones, but the active ones can be subdivided into the irrelevant and relevant ones. The latter group basically determines the network dynamics completely. Thus we might also suspect a difference in the attractor stabilities to be observed.

\paragraph*{}
There would also be the possibility to change the model into a more advanced one and redo the already done observations. We introduced logic as only being binary, but that is not the way it has to be. Ternary, quarternary or even $k$-valued logic systems would be thinkable and interesting to implement into the networks. The alternative to this would be to go over to probabilistic Boolean networks, for which we have found a good introduction by I. Shmulevich \cite{shmulevich2010probabilistic}. The author also writes about random Boolean networks, which was also useful for our own studies.

\section{Acknowledgements}
\paragraph*{}
First of all, I would like to thank my advisor Prof. Dr. Claudius Gros, who introduced me to my thesis's main topic and always provided useful hints on which way to go in order understand the attractors of random Boolean networks. Our regular discussions on all new results turned out to be very fruitful. I would also like to thank Prof. Dr. Barbara Drossel, who answered a question of mine. Her hint made me look at the cycle lengths more from a number theoretical perspective.
